% !TEX root =  master.tex
%      HYPERREF

%%%%%%%%%%%%%%%%%%%%%%%%%%%%%%%%%%%%%%%%%%%%%%%%%%%%%%%%%%
%	ANLEITUNG: 
%
% Passen Sie alle Stellen im Dokument an, die mit 
% @stud markiert sind
%
%%%%%%%%%%%%%%%%%%%%%%%%%%%%%%%%%%%%%%%%%%%%%%%%%%%%%%%%%%
\usepackage{listings}
\usepackage{makeidx}         % allows index generation
\usepackage{listings}	%Format Listings properly
\usepackage{lipsum}    %Blindtext
\usepackage{graphicx} % use various graphics formats
\usepackage[german]{varioref} 	% nicer references \vref
\usepackage{caption}	%better Captions
\usepackage{booktabs} %nicer Tabs
\usepackage{array}
\usepackage{chngcntr}
\usepackage[hidelinks=true]{hyperref} % keine roten Markierungen bei Links
\usepackage{fnpct} % Correct superscripts 
\usepackage[T1]{fontenc}
\usepackage[utf8]{inputenc}
\usepackage{calc} % Used for extra space below footsepline
\usepackage{acronym}
\usepackage{algorithm}
\usepackage{algpseudocode}
\usepackage{setspace}

%
% @stud
%
%	FONT SELECTION: Entweder 1) Latin Modern oder 2) Times / Helvetica
\usepackage{lmodern}             % 1) Latin modern font
%\usepackage{mathptmx}           % 2) Helvetica / Times New Roman fonts (2 lines)
%\usepackage[scaled=.92]{helvet} % 2) Helvetica / Times New Roman fonts (2 lines)

%
% @stud
%
%	LANGUAGE SETTINGS
\usepackage[ngerman]{babel} 	        % german language
\usepackage[german=quotes]{csquotes} 	% correct quoting using \enquote{}
%\usepackage[english]{babel}          % english language
%\usepackage{csquotes} 	              % correct quoting using \enquote{}

%
% @stud
%
% Uncomment the following lines to support hard URL breaks in bibliography 
%\apptocmd{\UrlBreaks}{\do\f\do\m}{}{}
%\setcounter{biburllcpenalty}{9000}% Kleinbuchstaben
%\setcounter{biburlucpenalty}{9000}% Großbuchstaben

%
% @stud
%
%	FOOTNOTES: Count footnotes over chapters
%1 \counterwithout{footnote}{chapter}

%	ACRONYMS
\makeatletter
\@ifpackagelater{acronym}{2015/03/20}
{\renewcommand*{\aclabelfont}[1]{\textbf{{\acsfont{#1}}}}}{}
\makeatother

%	LISTINGS
\renewcommand{\lstlistingname}{Quelltext} 
\renewcommand{\lstlistlistingname}{Quelltextverzeichnis}
\lstset{numbers=left,
	numberstyle=\tiny,
	captionpos=b,
	basicstyle=\ttfamily\small}

%	ALGORITHMS
\renewcommand{\listalgorithmname}{Algorithmenverzeichnis }
\floatname{algorithm}{Algorithmus}

%		PAGE HEADER / FOOTER
%	    Warning: There are some redefinitions throughout the master.tex-file!  DON'T CHANGE THESE REDEFINITIONS!
\RequirePackage[automark]{scrlayer-scrpage}
%alternatively with separation lines: \RequirePackage[automark,headsepline,footsepline]{scrlayer-scrpage}

%\renewcommand*{\pnumfont}{\upshape\sffamily}
%\renewcommand*{\headfont}{\upshape\sffamily}
%\renewcommand*{\footfont}{\upshape\sffamily}

\renewcommand{\chaptermarkformat}{}
\RedeclareSectionCommand[beforeskip=0pt]{chapter}
\clearscrheadfoot

%\ifoot[\rule{0pt}{\ht\strutbox+\dp\strutbox}DHBW Mannheim]{\rule{0pt}{\ht\strutbox+\dp\strutbox}DHBW Mannheim}
\ofoot[\rule{0pt}{\ht\strutbox+\dp\strutbox}\pagemark]{\rule{0pt}{\ht\strutbox+\dp\strutbox}\pagemark}
\ohead{\headmark}

\newcommand{\TitelDerArbeit}[1]{\def\DerTitelDerArbeit{#1}\hypersetup{pdftitle={#1}}}
#\newcommand{\AutorDerArbeit}[1]{\def\DerAutorDerArbeit{#1}\hypersetup{pdfauthor={#1}}}
\newcommand{\DerAutorDerArbeitEins}[1]{\def\DerAutorDerArbeitEins{#1}\hypersetup{pdfauthor={#1}}}
\newcommand{\DerAutorDerArbeitZwei}[1]{\def\DerAutorDerArbeitZwei{#1}\hypersetup{pdfauthor={#1}}}
\newcommand{\DerAutorDerArbeitDrei}[1]{\def\DerAutorDerArbeitDrei{#1}\hypersetup{pdfauthor={#1}}}
\newcommand{\DerAutorDerArbeitVier}[1]{\def\DerAutorDerArbeitVier{#1}\hypersetup{pdfauthor={#1}}}
\newcommand{\Firma}[1]{\def\DerNameDerFirma{#1}}
\newcommand{\Kurs}[1]{\def\DieKursbezeichnung{#1}}
\newcommand{\Abteilung}[1]{\def\DerNameDerAbteilung{#1}}
\newcommand{\Studiengangsleiter}[1]{\def\DerStudiengangsleiter{#1}}
\newcommand{\WissBetreuer}[1]{\def\DerWissBetreuer{#1}}

\newcommand{\Bearbeitungszeitraum}[1]{\def\DerBearbeitungszeitraum{#1}}
\newcommand{\Abgabedatum}[1]{\def\DasAbgabedatum{#1}}
\newcommand{\Matrikelnummer}[1]{\def\DieMatrikelnummer{#1}}
\newcommand{\Studienrichtung}[1]{\def\DieStudienrichtung{#1}}
\newcommand{\ArtDerArbeit}[1]{\def\DieArtDerArbeit{#1}}
\newcommand{\Literaturverzeichnis}{Literaturverzeichnis}

\newcommand{\settingBibFootnoteCite}{
	\setlength{\bibparsep}{\parskip}		  % Add some space between biblatex entries in the bibliography
	\addbibresource{bibliography.bib}	    % Add file bibliography.bib as biblatex resource
	\DefineBibliographyStrings{ngerman}{andothers = {{et\,al\adddot}},}
	\AdaptNoteOpt\footcite\multfootcite   % Will add  separators if footcite is called multiple consecutive times 
	\AdaptNoteOpt\autocite\multautocite   % Will add  separators if autocite is called multiple consecutive times
}

\newcommand{\setTitlepage}{
	% !TEX root =  master.tex
\begin{titlepage}
\begin{minipage}{\textwidth}
		\vspace{-2cm}
		\noindent \hfill \includegraphics{\imagedir/logo.jpg}
\end{minipage}
\vspace{1em}
%\sffamily
\begin{center}
	{\textsf{\large Duale Hochschule Baden-W\"urttemberg Mannheim}}\\[4em]
	{\textsf{\textbf{\large{\DieArtDerArbeit}arbeit}}}\\[6mm]
	{\textsf{\textbf{\Large{}\DerTitelDerArbeit}}} \\[1.5cm]
	{\textsf{\textbf{\large{}Studiengang Wirtschaftsinformatik}}\\[6mm]
	\textsf{\textbf{Studienrichtung \DieStudienrichtung}}}\vspace{10em}
	
	\begin{minipage}{\textwidth}
		\begin{tabbing}
		Wissenschaftlicher Betreuer: \hspace{0.85cm}\=\kill
		Verfasser und Matrikelnummer: \> \DerAutorDerArbeitEins \\[1.5mm]
		\> \DerAutorDerArbeitZwei \\[1.5mm]
		\> \DerAutorDerArbeitDrei \\[1.5mm]
		\> \DerAutorDerArbeitVier \\[1.5mm]
		Kurs: \> \DieKursbezeichnung \\[1.5mm]
		Studiengangsleiter: \> \DerStudiengangsleiter \\[1.5mm]
		Wissenschaftlicher Betreuer: \> \DerWissBetreuer \\[1.5mm]
		Bearbeitungszeitraum: \> \DerBearbeitungszeitraum\\[1.5mm]
%		alternativ:\\[1.5mm]
%		Eingereicht: \> \DasAbgabedatum	
		\end{tabbing}
	\end{minipage}
\end{center}
\end{titlepage}
	\pagenumbering{roman} % Römische Seitennummerierung
	\normalfont	
}

%
% @stud
%
\newcommand{\settingLists}{
	%	Inhaltsverzeichnis
	\tableofcontents
	%	Abbildungsverzeichnis
	\listoffigures
	%	Tabellenverzeichnis
	\listoftables
	%	Listingsverzeichnis / Quelltextverzeichnis
	\lstlistoflistings
	% Algorithmenverzeichnis
	\listofalgorithms
}

\newcommand{\initializeText}{
	\clearpage
	\ihead{\chaptername~\thechapter} % Neue Header-Definition
	\pagenumbering{arabic}           % Arabische Seitenzahlen
}

\newcommand{\initializeBibliography}{
	\ihead{}
	\printbibliography[title=\Literaturverzeichnis] 
	\cleardoublepage
}

\newcommand{\initializeAppendix}{
	\appendix
	\ihead{\appendixname~\thechapter}
}

