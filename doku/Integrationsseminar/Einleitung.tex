% !TEX root =  master.tex
\chapter{Einleitung}

In dieser wissenschaftlichen Arbeit wird sich mit dem Projekt der \enquote{Intelligente Spur- und Objekterkennung als Teil des autonomen Fahrens} beschäftigt.
Dieses Projekt ist Teil des Integrationsseminar an der \ac{DHBW} Mannheim im Wintersemester 2020/21.

Das Ziel des Projektes ist es ein System zu Entwickeln welches es ermöglicht Spuren und Objekten auf Bildern zu erkennen.
In der Planungsphase wurde sich dafür entschieden eine reale Umsetzung des Projektes mithilfe eines 1:10 ferngesteuertes Modellauto durchzuführen.
Um dies Umzusetzen muss das System so kompakt gebaut werden, dass ein RaspberryPi 3b+ die Ausführung des System bewältigen kann. 

Die verwendeten Algorithmen und Methoden belaufen sich auf \ac{YOLO} für die Objekterkennung. Für die Spurerkennung werden verschiedene Methoden der \ac{OpenCV} Programmierbibliothek verwendet.

\ac{YOLO} ist ein Algorithmus welcher eine abgewandelte Version eines \ac{CNN} verwendet um in einer Kürzeren zeit Objekte zu erkennen. 
Die genau Implementierung und Verwendung des \ac{YOLO} Algorithmus wird im Laufe der Arbeit detailliert beschrieben. 
Zusätzlich zu Objekterkennung wird hier auch die Entfernung zu den erkannten Objekten errechnet.

Die Spurerkennug ist basiert auf der \ac{OpenCV} Bibliothek. \ac{OpenCV} stellt viele verschiedene Algorithmen und Methoden zur Verfügung um Computer Vision umzusetzen. 
Welche Methoden und Algorithmen verwendet wurden wird in einem Kapitel genau beschrieben.

Die Vorgehensweise wie das Modellauto präpariert und modifiziert wurde um die Steuerung durch die Algorithmen zu ermöglichen wird in einem Abschließendem Kapitel beschrieben und dargestellt.
Dazu wird ebenfalls eine Anleitung bereitgestellt welche das verwenden der Systemem beschreibt.

